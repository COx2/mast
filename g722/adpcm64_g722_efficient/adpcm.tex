\documentstyle[times]{article}
\setlength{\parskip}{2ex}\setlength{\parindent}{0pt}
\renewcommand{\topfraction}{0.95}
\renewcommand{\textfraction}{0.05}
\setcounter{topnumber}{2}\setcounter{bottomnumber}{2}

\title{A fast program for Adaptive Differential Pulse Code Modulation}
\author{Chengxiang Lu and Alexander G. Hauptmann}

% Speech Group, School of Computer Science, Carnegie Mellon University, Pittsburgh, PA, USA
\begin{document}
\maketitle

% {\center{\bf ABSTRACT}\\}

Sound signals must be coded efficiently for many applications. 
One of these coding techniques is the 64 kbit/s audio coding system
which has been made a standard by the CCITT G.722.
The technique takes a 16 kHz (14 bit or 16 bit) input signal with a bandwidth
7kHz and encodes it to 64 kbits/s. 
The code was originally written by Milton Anderson (milton@thumper.bellcore.com)
 from BELLCORE, and has
been modified by the authors at the Speech Group, School of Computer Science, 
Carnegie Mellon University, to be fairly fast and efficient, 
while retaining high fidelity.

This program performs the 64kbit/s CCITT ADPCM CODEC fairly efficiently.
The program is written in ANSI "C" and has been optimized to run efficiently 
on many smaller workstations. We tested the program on a NeXT
workstation using a Motorola 68040 processor at 25 MHz and found it
to run in about .6 times real time.
The program is available by ftp from CMU, and may be used and distributed
freely, provided the copyright notices are maintained.

The Carnegie Mellon ADPCM program is Copyright (c) 1993
by Carnegie Mellon University. Use of this program, for any research or
commercial purpose, is completely unrestricted.  If you make use of or
redistribute this material, we would appreciate acknowlegement of its origin.

Feel free to 
contact the authors by e-mail (lu+@cs.cmu.edu or alex@cs.cmu.edu) 
for more information.
We would appreciate if you sent us any changes or improvements to the code
so we can further distribute them. If you register your name with us,
notice of all future updates will be automatically forwarded to you.

\section{Encoding:}

Encoding is done in the following steps:

\begin{enumerate}
\item Read 16kHz sampled 14 bit PCM values. 

\item Split the total frequency band of input signals into two 
  subbands -- (a Low-band and a High-band) by using QMF
(quadrature mirror filters).

\item Each band signal is encoded by ADPCM which 
   assigns 2 bits and 6 bits to each pair of high- and low-band signals, 
   respectively.
\end{enumerate}

\section{Decoding:}

[add decoder picture]

Decoding is done in the following steps:
\begin{enumerate}
\item Read encoded signals.

\item Each signal is decoded by ADPCM.

\item The decoded signals are subjected to a doubling of the sampling 
   frequency achieved by inserting zero samples. 

\item The two 16kHz sampled signals are fed into the receiver QMF and 
added together.

\end{enumerate}

For details about the basic procedure, refer to 
\cite{Papamichalis87,JayantandNoll84,IwadareandNishitani88}.

In the near future, we plan to test the ADPCM encoding with our recognition
system and report on accuracy decreases.


\bibliographystyle{plain}
\bibliography{adpcm}

\end{document}
